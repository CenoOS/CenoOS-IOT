% !Mode:: "TeX:UTF-8"
% !TEX root = tjumain.tex

\iffalse
\bibliography{reference/reference.bib} % 欺骗latextools获取bib文件
\fi

%%%%%%% 正文 %%%%%%%

\chapter{绪论}

\chapter{设计}



中文学位论文测试\cite{zhang2010tree}。

\subsection{参考文献标引}

一只敏捷的棕色狐狸跳过那只懒惰的狗\cite{deng:01a}。

\chapter{实现}

\section{BOOT}
\subsection{Interrupt Vector Table}
首先我们从中断向量表(既中断服务程序入口地址)开始,当中断或异常发生的时候, CPU自动将PC指向一个特定的地址, 这个地址就是中断向量表。 Arm的中断向量表一般位于内存0x00000000\textasciitilde0x0000001c处, 结构如下:
\begin{table}[htbp]
    \caption{ Arm 中断向量表}\label{tab:table1}
    \vspace{0.5em}\centering\wuhao
    \begin{tabular}{ccccc}
    \toprule[1.5pt]
    地址 & 中断 \\
    \midrule[1pt]

    0x00000040 & 269.8 \\
    0x0000003C & 421.0 \\
    0x00000038 & 640.2 \\
    0x00000034 & 269.8 \\
    0x00000030 & 421.0 \\
    0x0000002C & 640.2 \\
    0x00000028 & 269.8 \\
    0x00000024 & 421.0 \\
    0x00000020 & 640.2 \\
    0x0000001C & 269.8 \\
    0x00000018 & 421.0 \\
    0x00000014 & 421.0 \\
    0x00000010 & 421.0 \\
    0x0000000C & 421.0 \\
    0x00000008 & 421.0 \\
    0x00000004 & Reset\_Handler \\
    0x00000000 & Top\_Of\_Stack \\
    \bottomrule[1.5pt]
    \end{tabular}
    \vspace{\baselineskip}
    \end{table}
    
\subsection{Start Up}


\begin{lstlisting}
int main(int argc, char ** argv) {
    printf("Hello world!\n");
    return 0;
}
\end{lstlisting}

\section{TCB}
\section{Interrupt}
\section{SysTick  PendSV}
\section{Context Switch}
\section{Thread Scheduling}

\section{Critical Section}
\section{Semphore}
\section{Mutex}
\section{MQ}


%%%%%%% 结论 %%%%%%%

\addcontentsline{toc}{chapter}{结\quad 论} %添加到目录中

\chapter*{结\quad 论}

得出结论,楼主傻逼。
