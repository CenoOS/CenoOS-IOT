% !Mode:: "TeX:UTF-8"

\titlecontents{chapter}[2em]{\vspace{.5\baselineskip}\xiaosan\song}%
             {\prechaptername\CJKnumber{\thecontentslabel}\postchaptername\qquad}{} %
             {}             % 设置该选项为空是为了不让目录中显示页码
\addcontentsline{toc}{chapter}{外文资料}
%\setcounter{page}{1}       % 如果需要从该页开始从 1 开始编页,则取消该注释
\markboth{外文资料}{外文资料}
\chapter*{外文资料}
\subsection{Introduction}
This paper introduces LiteOS, an operating system that
provides Unix-like abstractions to wireless sensor networks.
LiteOS maps a sensor network into a UNIX-like file system,
and supports extremely resource-constrained nodes such as
MicaZ. It supports C programming natively, and allows online debugging to locate application bugs. We believe that
such an operating system could potentially expand the circle of sensor network application developers by providing
a familiar programming environment. While TinyOS and
its extensions have significantly improved programmability
of mote-class embedded devices via a robust, modular environment, NesC and the event-based programming model
introduce a learning curve for most developers outside the
sensor networks circle. The purpose of LiteOS is to significantly reduce such a learning curve. This philosophy is
the operating system equivalent of network directions taken
by companies such as Arch Rock [1] (that superimposes a
familiar IP space on mote platforms to reduce the learning
curve of network programming and management).

Our key contribution is to present a familiar, Unix-like
abstraction for wireless sensor networks by leveraging the
likely existing knowledge that common system programmers (outside the current sensor network community) already have: Unix, threads, and C. By mapping sensor networks to file directories, it allows applying user-friendly
operations, such as file directory commands, to sensor networks, therefore reducing the learning curve for operating
and programming sensor networks.

LiteOS differs from both current sensor network operating systems and more conventional embedded operating
systems. Compared to the former category, such as TinyOS,
LiteOS provides a more familiar environment to the user. Its
features are either not available in existing sensor network
operating systems, such as the shell and the hierarchical file
system, or are only partially supported. Compared to the
latter category (conventional embedded operating systems),
such as VxWorks [29], eCos [2], embedded Linux, and
Windows CE, LiteOS has a much smaller code footprint,
running on platforms such as MicaZ, with an 8MHz CPU,
128K bytes of program flash, and 4K bytes of RAM. Embedded operating systems, such as VxWorks, require more
computation power (e.g., ARM-based or XScale-based processors) and more RAM (at least tens of KBytes), and thus
cannot be easily ported to MicaZ-class hardware platforms
(such as MicaZ, Tmote, and Telos).
A possible counter-argument to our investment in asmall-footprint UNIX-like operating system is that, in the
near future, Moore’s law will make it possible for conventional Linux and embedded operating systems to run on
motes. For example, the recent iMote2 [6] platform by
CrossBow features an XScale processor that supports embedded Linux. Sun and Intel also demonstrated more powerful sensor network hardware platforms [3, 24]. While it is
true that more resources will be available within the current
mote form factor, Moore’s law can also be harvested by decreasing the form factor while keeping resources constant.
For example, the current MicaZ form factor is far from adequate for wearable computing applications. Wearable body
networks can have a significant future impact on healthcare, leisure, and social applications if sensor nodes could
be made small enough to be unobtrusively embedded in attire and personal effects. These applications will drive the
need for small-footprint operating systems and middleware
as computation migrates to nodes that are smaller, cheaper,
and more power-efficient. This paper serves as a proof of
concept by pushing the envelope within the constraints of a
current device; namely, the MicaZ motes.



\subsection{LiteShell Subsystem}
The LiteShell subsystem provides Unix-like commandline interface to sensor nodes. This shell runs on the base
station PC side. Therefore, it is a front-end that interacts
with the user. The motes do not maintain command-specific
state, and only respond to translated messages (represented
by compressed tokens) from the shell, which are sufficiently
simple to parse. Such an asymmetric design choice not only
significantly reduces the code footprint of the LiteOS kernel
that runs on motes, but also allows us to easily extend the
shell with more complicated commands, such as authentication and security.


\subsubsection{File Operation Commands}
File commands generally maintain their Unix meanings.
For example, the ls command lists directory contents. It
supports a -l option to display detailed file information, such
as type, size, and protection. To reduce system overhead,
LiteOS does not provide any time synchronization service,
which is not needed by every application. Hence, there is
no time information listed. As an example, a ls -l command
may return the following:
\begin{lstlisting}[language={[ANSI]C},keywordstyle=\color{blue!70},commentstyle=\color{red!50!green!50!blue!50},frame=shadowbox, rulesepcolor=\color{red!20!green!20!blue!20}]
$ ls -l
Name Type Size Protection
usrfile file 100 rwxrwxrwx
usrdir dir --- rwxrwx---
\end{lstlisting}

In this example, there are two files in the current directory (a directory is also a file): usrfile and usrdir. LiteOS
enforces a simple multilevel access control scheme. All
users are classified into three levels, from 0 to 2, and 2 is
the highest level. Each level is represented by three bits,
stored on sensor nodes. For instance, the usrdir directory
can be read or written by users with levels 1 and 2.

Once sensor nodes are mounted, a user uses the above
commands to navigate the different directories (nodes) as if
they are local. Some common tasks can be greatly simplified. For example, by using the cp command, a user can
either copy a file from the base to a node to achieve wireless download, or from a node to the base to retrieve data
results. The remaining file operation commands are intuitive. Since LiteFS supports a hierarchical file system, it
provides mkdir, rm and cd commands.

\subsubsection{Process Operation Commands}
LiteOS has a multithreaded kernel to run applications as
threads concurrently. LiteShell provides three commands
to control thread behavior: ps, exec, and kill. We illustrate
these commands through an application called Blink, which
blinks LEDs periodically. Suppose that this application has
been compiled into a binary file called Blink.lhex1, and is
located under the C drive of the user’s laptop. To install it
on a node named node101 (that maps to a directory with the
same name) in a sensor network named sn01, the user may
use the following commands:
\begin{lstlisting}[language={[ANSI]C},keywordstyle=\color{blue!70},commentstyle=\color{red!50!green!50!blue!50},frame=shadowbox, rulesepcolor=\color{red!20!green!20!blue!20}]
$ pwd
Current directory is /sn01/node101/apps
$ cp /c/Blink.lhex Blink.lhex
Copy complete
$ exec Blink.lhex
File Blink.lhex successfully started
$ ps
Name State
Blink Sleep
\end{lstlisting}

As illustrated in this example, we first copy an application, Blink.lhex, into the /apps directory, so that it is stored
in the LiteFS file system. We then use the exec command to
start this application. The implementation of exec is as follows. The processor architecture of Atmega128 follows the
Harvard architecture, which provides a separate program
space (flash) from its data space (RAM). Only instructions
that have been programmed into the program space can be
executed. Hence, LiteOS reprograms part of the flash to run
the application.

Once the Blink application is started, the user may view
its thread information using the ps command. Finally, the
kill command is used to terminate threads.

\subsubsection{Debugging Commands}
We now describe the debugging commands. Eight commands are provided, including those for setting up the debugging environment (debug), watching and setting variables (list, print, and set), adding/removing breakpoints
(breakpoint and continue), and application checkpoints
(snapshot and restore). Note that all debugging commands
keep information on the front-end, i.e., the PC side. In
fact, there is no debugging state stored on the mote, which
means that there is no limit on the maximum number of
variables (or the size of variables) can be watched, or how
many breakpoints can be added. We now briefly explain
these commands. Detailed documentation of these commands can be found in the LiteOS manual.

The user first invokes the debug command to initiate
the environment. This command takes the source code directory of the application as its parameter. For example,
if supplied with the kernel source code location, it allows
debugging the kernel itself. Once invoked, this command
parses the source code as well as the generated assembly
to gather necessary information, such as memory locations
of variables. Such information is then used by other debugging commands for diagnosis purposes. For instance,
it is used by the command list to display the current variables and their sizes, commands print and set to watch
and change variable values, and commands breakpoint and
continue to add and remove breakpoints. Once a breakpoint
is added, the command ps tells whether a thread has reached
the breakpoint.

We now explain the commands snapshot and restore.
Snapshot allows adding a checkpoint to an active thread,
by exporting all its memory information, including variable
values, stack, and the program counter, to an external file.
Restore, on the other hand, allows importing such memory information from a previously generated file, essentially
restoring a thread to a previous state. Combined use of these
two commands allows replaying part of an application by
rewinding it, and is particularly useful for locating unexpected bugs.

\subsubsection{Environment Commands}
The next four commands support environment management: history for displaying previously used commands,
who for showing the current user, man for command references, and echo for displaying strings. The meanings of
these commands are similar to their Unix counterparts.

\subsubsection{Device Commands}
The LiteOS shell provides an easy way to interact with the
sensors. Every time the file system is initialized, a directory
dev is created, which contains files that map to actual device
drivers. On MicaZ, the dev directory contains the following
files:
\begin{lstlisting}[language={[ANSI]C},keywordstyle=\color{blue!70},commentstyle=\color{red!50!green!50!blue!50},frame=shadowbox, rulesepcolor=\color{red!20!green!20!blue!20}]
$ls
led, light, temp, magnet, accel, radio
\end{lstlisting}

In this directory, led refers to the LED device. There are
four sensors, light, temperature, magnetic, and accelerator,
respectively. There is also the radio device, which sends and
receives packets. An example of reading 100 data samples
from the light sensor at a frequency of 50 milliseconds is
written as follows, where the first parameter is the frequency
and the second parameter is the number of readings.

\begin{lstlisting}[language={[ANSI]C},keywordstyle=\color{blue!70},commentstyle=\color{red!50!green!50!blue!50},frame=shadowbox, rulesepcolor=\color{red!20!green!20!blue!20}]
./light 50 100
\end{lstlisting}   


\subsection{LiteFS Subsystem}
\subsubsection{Implementation of LiteFS}
Figure 3 shows the architecture of LiteFS, which is partitioned into three modules. It uses RAM to keep opened files
and the allocation information of EEPROM and the data
flash in the first module, uses EEPROM to keep hierarchical
directory information in the second, and uses the data flash
to store files in the third. Just like Unix, files in LiteFS represent different entities, such as data, application binaries,
and device drivers. A variety of device driver files, including radio, sensor, and LED, are supported. Their read/write
operations are mapped to real hardware operations. For example, writing a message to the radio file (either through the
shell by a user or through a system call by an application)
maps to broadcasting this message.

In RAM, our current version of LiteOS supports eight
file handles (this number is adjustable according to application needs), where each handle occupies eight bytes. Hence,
at most eight files can be opened simultaneously. LiteFS
uses two bit vectors to keep track of EEPROM/flash allocation, one with 8 bytes for EEPROM, the other with 32 bytes
for the serial flash. A total of 104 bytes of RAM are used to
support these bit vectors.

In EEPROM, each file is represented as a 32-byte control block. LiteFS currently uses 2080 bytes of the 4096
bytes available in EEPROM to store hierarchical directories, while the remaining EEPROM is available for other
needs. These 2080 bytes are partitioned into 65 blocks. The
first block is the root block, which is initialized every time
the file system is formatted. The other 64 blocks are either
directory blocks (specified with D) or file blocks (specified
with F) according to application needs. Just like Unix, files
represent data, binary applications, and device drivers.

We follow three design choices in LiteFS. First, the maximal length of a file name is 12 bytes. Therefore, the Eight
Dot Three naming system is supported. Second, a file control block addresses at most ten logical flash pages. Each
page holds 2K bytes of data (or 8 physical flash pages). If a
file occupies more than 20K bytes, LiteFS allocates another
control block for this file, and stores the address of the new
block in the old one. Third, all control blocks reside in EEPROM rather than in RAM.